\section{Unicité de la grille finale et des cases effondrées [Florent]}

Soit $\rightarrow$ une relation binaire sur un ensemble $E$. 

Pour $T,T'\in E$, on définit $\rightarrow^n$ pour $n\in\mathbb{N}$ par récurrence :
\begin{itemize}
\item $T\rightarrow^0 T' \Leftrightarrow T=T'$
\item  $T\rightarrow^{n+1} T' \Leftrightarrow \exists T_1\in E,\ T\rightarrow^n T_1\rightarrow T'$
\end{itemize} 

Et enfin, on définit : $T\rightarrow^* T' \Leftrightarrow \exists n\in\mathbb{N},T\rightarrow^nT'$.

Soit $T,T'\in\mathcal{M}_{m,n}(\mb{N})$ %(je reprends les notations d'Hugo).

Nous définissons une relation binaire $\rightarrow_e$ entre les configurations de la grille comme suit : 
$T\rightarrow_e T'\Leftrightarrow \exists(i,j),T\xrightarrow{(i,j)}T'$.




\begin{lem}
$\rightarrow_e$ est localement confluente, i.e. pour $T, T_1, T_2\in\mathcal{M}_{m,n}(\mb{N})$ : 
$$\left.\begin{array}{ll}T\rightarrow_eT_1\\T\rightarrow_eT_2\end{array}\right\}\Rightarrow \exists T'\in\mathcal{M}_{m,n}(\mb{N}),\ T_1\rightarrow_e^*T' \mathrm{\ et\ }T_2\rightarrow_e^*T'$$
\end{lem}
\begin{proof}
Soient $T, T_1, T_2\in\mathcal{M}_{m,n}(\mb{N})$ tels que $T\rightarrow_eT_1$ et $T\rightarrow_eT_2$.

1er cas : $T_1=T_2$. Alors il suffit de prendre $T'=T_2=T_1$, et ne pas effectuer de transition.

2e cas : $T_1\not = T_2$. Soit $(i_1, j_1)$ (respectivement $(i_2, j_2)$) la case de $T$ qui a été effondrée pour obtenir $T_1$ (respectivement $T_2$).

On sait donc que $T(i_1, j_1)\geqslant 4$ et que $T(i_2, j_2)\geqslant 4$.

Or, on remarque que lors d'une transition, seule la case qui s'effondre voit son nombre de grains baisser. Par conséquent, comme $(i_1, j_1)\not = (i_2, j_2)$ (car $T_1\not=T_2$), on en déduit que $T_2(i_1, j_1)\geqslant 4$ et que $T_1(i_2, j_2)\geqslant 4$. On peut donc les effondrer : 
$$\exists T', T''\in\mathcal{M}_{m,n}(\mb{N}), T_1\xrightarrow{(i_2,j_2)}T',\ T_2\xrightarrow{(i_1,j_1)}T'' $$

Montrons que $T'=T''$, et on aura démontré la confluence locale. Pour cela, regardons comment a été obtenu $T'$ à partir de $T$ : $T\xrightarrow{(i_1,j_1)}T_1\xrightarrow{(i_2,j_2)}T'$
\begin{enumerate}
\item la case $(i_1,j_1)$ a été décrémentée de 4, les cases adjacentes à $(i_1,j_1)$ ont été incrémentées de 1.
\item puis la case $(i_2,j_2)$ a été décrémentée de 4, les cases adjacentes à $(i_2,j_2)$ ont été incrémentées de 1.
\end{enumerate}

Au plus 10 cases sont affectées par ces transitions. On remarque que ce sont les mêmes cases qui sont affectées lors des transitions $T\xrightarrow{(i_2,j_2)}T_2\xrightarrow{(i_1,j_1)}T''$.

Or l'addition est associative et commutative sur les entiers relatifs, et les deux cases qui perdent des grains ont dès la configuration initiale un nombre de grains $\geqslant 4$, et ne sont effondrées qu'une fois chacunes ; par conséquent l'ordre des opérations ne compte pas dans cette suite de deux transitions. Donc en effondrant à partir de $T$ d'abord la case $(i_2,j_2)$, puis la case $(i_1,j_1)$, on obtient toujours $T'$.

Et comme $T\xrightarrow{(i_2,j_2)}T_2\xrightarrow{(i_1,j_1)}T''$, on en conclut que $T'=T''$.

\end{proof}

$T\in\mathcal{M}_{m,n}(\mb{N})$ est dit \textbf{normal} s'il n'existe pas de $T'\in\mathcal{M}_{m,n}(\mb{N})$ tel que $T\rightarrow_eT'$, autrement dit si toutes ses cases ont un nombre de grains inférieurs à 4 strictement, autrement dit si $T\in\mathcal{M}_{m,n}(\{0,1,2,3\})$.

\begin{lem}
$\rightarrow_e$ est fortement normalisante, i.e. pour tout $T\in\mathcal{M}_{m,n}(\mb{N})$, il existe $T'\in\mathcal{M}_{m,n}(\mb{N})$ tel que $T\rightarrow_e^*T'$ et $T'$ est normal.
\end{lem}
\begin{proof}
Si $\rightarrow_e$ n'était pas fortement normalisante, il existerait une grille qui admettrait une suite de transitions infinies, donc le procédé du tas de sable ne terminerait pas, ce qui est impossible, d'après les preuves de terminaison ci-dessus.
\end{proof}

\begin{lem} [\bsc{Lemme de Newman}]
Une relation binaire $\rightarrow$ sur un ensemble $E$ localement confluente et fortement normalisante est confluente, i.e. pour $T, T_1, T_2\in E$ : 
$$\left.\begin{array}{ll}T\rightarrow^*T_1\\T\rightarrow^*T_2\end{array}\right\}\Rightarrow \exists T'\in E,\ T_1\rightarrow^*T' \mathrm{\ et\ }T_2\rightarrow^*T'$$
\end{lem}
\begin{proof}
Voir J. Goubault.
\end{proof}

\begin{theo}
La grille finale après l'exécution de l'algorithme du tas de sable sur une grille donnée est unique.
\end{theo}
\begin{proof}
Soit $T\in\mathcal{M}_{m,n}(\mb{N})$. Soient $T'$ et $T''$ deux grilles finales obtenues avec deux exécutions de l'algorithme du tas de sable.

Alors, avec notre formalisme, $T\rightarrow_e^*T'$ et $T\rightarrow_e^*T''$, et $T'$ et $T''$ sont normales.

Or, $\rightarrow_e$ est localement confluente et fortement normalisante, donc d'après le lemme de Newman, $\rightarrow_e$ est confluente.

Donc il existe $T'''$ tel que $T'\rightarrow_e^*T'''$ et  $T''\rightarrow_e^*T'''$. Mais comme $T'$ est normale, $T'=T'''$ et de même, $T''=T'''$. Donc $T'=T''$.

\end{proof}

\begin{rem} Nous avons démontré dans la preuve de confluence locale que si $T\xrightarrow{(i_1,j_1)}T_1\xrightarrow{(i_2,j_2)}T'$, et si les cases $(i_1,j_1)$ et $(i_2, j_2)$ sont effondrables depuis $T$, alors on a aussi $T\xrightarrow{(i_2,j_2)}T_2\xrightarrow{(i_1,j_1)}T'$.\end{rem}

\begin{coro} Soit une suite de transitions $T\xrightarrow{(i_1,j_1)}T_1\xrightarrow{(i_2,j_2)}T_2\cdots\xrightarrow{(i_n,j_n)}T_n$, telle que la case $(i_n, j_n)$ ne soit effondrée qu'une seule fois, et soit effondrable dans $T$. 

Alors, il existe $T'_1, \ldots, T'_{n-1}$ tels que : 
$$T\xrightarrow{(i_n,j_n)}T'_1\xrightarrow{(i_1,j_1)}T'_2\cdots\xrightarrow{(i_{n-2},j_{n-2})}T'_{n-1}\xrightarrow{(i_{n-1},j_{n-1})}T_n$$
 \end{coro}
 
\begin{rem}L'important est de voir que : 
\begin{enumerate}
\item La grille de début et de fin ne changent pas.
\item Les cases effondrées sont les mêmes.
\item L'ordre des cases effondrées a changé : la dernière transition a été placée au début.
\end{enumerate}
\end{rem}
 \begin{proof}
 Par récurrence sur $n$.
 
 Pour $n=1$, le théorème est immédiat.
 
 Le cas $n=2$ découle de l'avant-dernière remarque.
 
 Pour le cas $n+1$ : supposons qu'on ait $T\xrightarrow{(i_1,j_1)}T_1\xrightarrow{(i_2,j_2)}T_2\cdots\xrightarrow{(i_{n+1},j_{n+1})}T_{n+1}$.
 
 En particulier, nous avons $T_{n-1}\xrightarrow{(i_{n},j_{n})}T_n\xrightarrow{(i_{n+1},j_{n+1})}T_{n+1}$.
 
 Comme par hypothèse, la case $(i_{n+1},j_{n+1})$ est effondrable depuis $T$, et n'est jamais effondrée avant la dernière transition, on en conclut que $T_k(i_{n+1},j_{n+1})\geqslant 4$ pour $1\leqslant k \leqslant n$, i.e. cette case est effondrable sur toutes les grilles intermédiaires, donc depuis $T_{n-1}$.
 
 De plus, la case $(i_{n},j_{n})$ est effondrable depuis $T_{n-1}$. Par conséquent, en appliquant la remarque (ou le cas $n=2$), on en conclut qu'il existe $T'_n$ tel que : 
 $$T_{n-1}\xrightarrow{(i_{n+1},j_{n+1})}T'_n\xrightarrow{(i_{n},j_{n})}T_{n+1}$$
 
 En appliquant l'hypothèse de récurrence à $T\xrightarrow{(i_1,j_1)}T_1\xrightarrow{(i_2,j_2)}T_2\cdots\xrightarrow{(i_{n-1},j_{n-1})}T_{n-1}\xrightarrow{(i_{n+1},j_{n+1})}T'_n$, on obtient l'existence de $T'_1, \ldots, T'_{n-1}$ tels que : 
$$T\xrightarrow{(i_{n+1},j_{n+1})}T'_1\xrightarrow{(i_1,j_1)}T'_2\cdots\xrightarrow{(i_{n-2},j_{n-2})}T'_{n-1}\xrightarrow{(i_{n-1},j_{n-1})}T'_n$$

Et finalement, en recollant : 

$$T\xrightarrow{(i_{n+1},j_{n+1})}T'_1\xrightarrow{(i_1,j_1)}T'_2\cdots\xrightarrow{(i_{n-2},j_{n-2})}T'_{n-1}\xrightarrow{(i_{n-1},j_{n-1})}T'_n\xrightarrow{(i_{n},j_{n})}T_{n+1}$$
 
  \end{proof}

\begin{theo}
Soient deux suites d'exécution du tas de sable : $T\xrightarrow{(i_1,j_1)}T_1\xrightarrow{(i_2,j_2)}T_2\cdots\xrightarrow{(i_n,j_n)}T_f$ et $T\xrightarrow{(i'_1,j'_1)}T'_1\xrightarrow{(i'_2,j'_2)}T'_2\cdots\xrightarrow{(i'_{n'},j'_{n'})}T_f$, avec $T_f$ normal. 

Alors $\{\!\!\{(i_1, j_1), \ldots, (i_n, j_n) \}\!\!\}=\{\!\!\{(i'_1, j'_1), \ldots, (i'_{n'}, j'_{n'}) \}\!\!\}$ (égalité des multiensembles, c'est-à-dire que les cases effondrées sont les mêmes, et sont effondrées le même nombre de fois. En particulier, $n=n'$).
\end{theo}

\begin{proof}
On effectue une récurrence sur la longueur $L$ de la plus petite séquence. Supposons sans perte de généralité que la plus petite séquence est la première séquence. 

Si $L=1$, alors on a $T\xrightarrow{(i_1,j_1)}T_f$ et $T\xrightarrow{(i'_1,j'_1)}T'_1\xrightarrow{(i'_2,j'_2)}T'_2\cdots\xrightarrow{(i'_{n'},j'_{n'})}T_f$. 

Comme $T_f$ est normal, au moins toutes les cases de $T$ qui étaient effondrables ont été effondrées. Donc la case $(i'_1, j'_1)$ est effondrée à un moment dans la première séquence. Donc $(i_1, j_1)=(i'_1, j'_1)$. Donc $T'_1=T_f$ et est donc normal, donc $n'=1$.

Pour $L\geqslant2$, on va utiliser le corollaire. On part de deux séquences $T\xrightarrow{(i_1,j_1)}T_1\xrightarrow{(i_2,j_2)}T_2\cdots\xrightarrow{(i_n,j_n)}T_f$ et $T\xrightarrow{(i'_1,j'_1)}T'_1\xrightarrow{(i'_2,j'_2)}T'_2\cdots\xrightarrow{(i'_{n'},j'_{n'})}T_f$, et on suppose $L=n\leqslant n'$.

Comme $T_f$ est normale, au moins toutes les cases susceptibles d'être effondrées dès l'état $T$ l'ont été pour atteindre $T_f$. Donc d'après la première suite de transitions, la case $(i_1, j_1)$ est effondrée à un moment dans la seconde séquence.

Soit $k$ le plus petit entier tel que $(i_1, j_1)=(i'_k, j'_k)$.

Alors dans la suite de transitions $T\xrightarrow{(i'_1,j'_1)}T'_1\xrightarrow{(i'_2,j'_2)}T'_2\cdots\xrightarrow{(i'_{k},j'_{k})}T'_{k+1}$, la case $(i_1, j_1)=(i'_k, j'_k)$ n'est effondrée qu'une seule fois, et est effondrable dans $T$. D'après le corollaire, il existe $Q'_1, \ldots, Q'_{k}$ tels que :
$$T\xrightarrow{(i_1,j_1)}Q'_1\xrightarrow{(i'_1,j'_1)}Q'_2\cdots\xrightarrow{(i'_{k-1},j'_{k-1})}T'_{k+1}$$

Mais alors, en recollant, on a : 

$$T\xrightarrow{(i_1,j_1)}T_1\xrightarrow{(i_2,j_2)}T_2\cdots\xrightarrow{(i_n,j_n)}T_f$$
$$T\xrightarrow{(i_1,j_1)}Q'_1\xrightarrow{(i'_1,j'_1)}Q'_2\cdots\xrightarrow{(i'_{k-1},j'_{k-1})}T'_{k+1}\xrightarrow{(i'_{k+1},j'_{k+1})}\cdots\xrightarrow{(i'_{n'},j'_{n'})}T_f$$

On en déduit que $T_1=Q'_1$, et donc que l'on dispose maintenant des deux séquences de longueur plus petites : 

$$T_1\xrightarrow{(i_2,j_2)}T_2\cdots\xrightarrow{(i_n,j_n)}T_f$$
$$T_1\xrightarrow{(i'_1,j'_1)}Q'_2\cdots\xrightarrow{(i'_{k-1},j'_{k-1})}T'_{k+1}\xrightarrow{(i'_{k+1},j'_{k+1})}\cdots\xrightarrow{(i'_{n'},j'_{n'})}T_f$$

avec $T_f$ qui est normal. Par hypothèse de récurrence,\\
$\{\!\!\{(i_2, j_2), \ldots, (i_n, j_n) \}\!\!\}=\{\!\!\{(i'_1, j'_1), \ldots, (i'_{k-1},j'_{k-1}), (i'_{k+1},j'_{k+1}), \ldots (i'_{n'}, j'_{n'}) \}\!\!\}$, et comme $(i_1, j_1)=(i'_k, j'_k)$, on a finalement : 

$$\{\!\!\{(i_1, j_1), \ldots, (i_n, j_n) \}\!\!\}=\{\!\!\{(i'_1, j'_1), \ldots, (i'_{n'}, j'_{n'}) \}\!\!\}$$


\end{proof}


%\begin{rem}
%Je pense que le résultat précédent tient toujours même quand $T_f$ n'est pas normal.
%\end{rem}
\begin{coro}
Quelle que soit l'exécution de l'algorithme du tas de sable sur une grille $T$, la grille finale est unique, et la liste des cases effondrées est unique à l'ordre près.
\end{coro}

\begin{theo}
Soient deux suites d'exécution du tas de sable : $T\xrightarrow{(i_1,j_1)}T_1\xrightarrow{(i_2,j_2)}T_2\cdots\xrightarrow{(i_n,j_n)}T'$ et $T\xrightarrow{(i'_1,j'_1)}T'_1\xrightarrow{(i'_2,j'_2)}T'_2\cdots\xrightarrow{(i'_{n'},j'_{n'})}T'$, avec $T'$ \textbf{quelconque}. 

Alors $\{\!\!\{(i_1, j_1), \ldots, (i_n, j_n) \}\!\!\}=\{\!\!\{(i'_1, j'_1), \ldots, (i'_{n'}, j'_{n'}) \}\!\!\}$.
\end{theo}
\begin{proof}
La différence avec la proposition précédente est que $T'$ n'est pas forcément normal.

Comme on a la forte normalisation, on sait qu'il existe une suite d'exécution du tas de sable : $T'\xrightarrow{(i''_1,j''_1)}T''_1\xrightarrow{(i''_2,j''_2)}T''_2\cdots\xrightarrow{(i''_m,j''_m)}T_f$ avec $T_f$ normal.

On a donc les deux séquences suivantes : 
$$T\xrightarrow{(i_1,j_1)}T_1\xrightarrow{(i_2,j_2)}T_2\cdots\xrightarrow{(i_n,j_n)}T'\xrightarrow{(i''_1,j''_1)}T''_1\xrightarrow{(i''_2,j''_2)}T''_2\cdots\xrightarrow{(i''_m,j''_m)}T_f$$
$$T\xrightarrow{(i'_1,j'_1)}T'_1\xrightarrow{(i'_2,j'_2)}T'_2\cdots\xrightarrow{(i'_{n'},j'_{n'})}T'\xrightarrow{(i''_1,j''_1)}T''_1\xrightarrow{(i''_2,j''_2)}T''_2\cdots\xrightarrow{(i''_m,j''_m)}T_f$$ 
avec $T_f$ normal.

D'après la proposition précédente, on a $\{\!\!\{(i_1, j_1), \ldots, (i_n, j_n), (i''_1,j''_1), \ldots, (i''_m,j''_m) \}\!\!\}=\{\!\!\{(i'_1, j'_1), \ldots, (i'_{n'}, j'_{n'}), (i''_1,j''_1), \ldots, (i''_m,j''_m) \}\!\!\}$.

Et donc $\{\!\!\{(i_1, j_1), \ldots, (i_n, j_n) \}\!\!\}=\{\!\!\{(i'_1, j'_1), \ldots, (i'_{n'}, j'_{n'}) \}\!\!\}$.
\end{proof}


\medbreak
\medbreak
\medbreak

