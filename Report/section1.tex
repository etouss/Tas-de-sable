\section{Présentation d'un modèle tas de sable}
Cette section va poser les notations et les définitions relatives au modèle tas de sable, qui seront utilisées dans le reste de ce document.

%Soit $T$ une matrice de taille $n\times m$ remplie d'entiers naturels ($T(i,j)$ désignant l'entier dans la ligne i et la colonne j de $T$). Informellement, $T$ représente un tas de sable, séparé en $n*m$ piles contenant chacune un certain nombre de grains de sable.

\begin{definition}[\bsc{Tas de sable fini}]

Un tas de sable fini est une matrice dont tous les coefficients sont des entiers naturels.
Dans le reste du document, T(i,j) représentera le nombre de grains de sable de la case de coordonnée (i,j) ;
m, resp. n  représentera le nombre de lignes, resp. colonnes de la matrice.\\
On notera $|T|= \sum\limits_{\substack{1 \leq i\leq m\\ 1 \leq j\leq n}} T(i,j)$ le nombre total de grains de sable de T.
\end{definition}

Informellement, les cases d'un tas de sable représentent des piles de grains de sable.
On définit ensuite une opération transformant un tas de sable en un autre : l'effondrement.

\begin{definition}[\bsc{Effondrement}]

On note $T\xrightarrow{(i,j)}T'$ si l'effondrement de la case de coordonnées $(i,j)$ transforme le tas de sable $T$ en le tas de sable $T'$. C'est-à-dire si $\forall u\in\llbracket 1,m\rrbracket, \forall v\in \llbracket 1,n\rrbracket ,T(u,v)=T'(u,v)$ sauf dans les cas suivants :

\begin{itemize}
\item $T'(i,j)=T(i,j)-4$.
\item Si $i\neq 1$ alors $T'(i-1,j)=T(i-1,j)+1$.
\item Si $i\neq m$ alors $T'(i+1,j)=T(i+1,j)+1$.
\item Si $j\neq 1$ alors $T'(i,j-1)=T(i,j-1)+1$.
\item Si $j\neq n$ alors $T'(i,j+1)=T(i,j+1)+1$.
\end{itemize}
\textbf{Restriction} : un effondrement ne peut pas avoir lieu sur une case contenant strictement moins de 4 grains. Ceci assure la positivité des coefficients de la matrice $T'$.

On dira qu'une case $(i,j)$ est \textbf{effondrable} dans $T$ si $T(i,j)\geqslant4$.
\end{definition}



\textbf{Exemple d'effondrement:}
\begin{center}
\begin{tabular}{|c|c|}
\hline
1&4\\
\hline
2&\textbf{\textcolor{red}{6}}\\
\hline
\end{tabular}
\quad
$\xrightarrow{(2,2)}$
\quad
\begin{tabular}{|c|c|}
\hline
1&5\\
\hline
3&2\\
\hline
\end{tabular}
\end{center}

%On s'intéresse maintenant à l'évolution d'un tas de sable subissant de multiples effondrements successifs.
On présente ici des lemmes qui découlent de la définition, qui seront utiles plus tard.
\begin{lem}
\label{lemmedecroissance}
Soit $T,T'\in\mathcal{M}_{m,n}(\mb{N})$ tels que $T\xrightarrow{(i,j)}T'$. Alors $|T|\geq |T'|$. 

De plus $|T|=|T'|$ si et seulement si $i\neq 1$ et $i\neq m$ et $j\neq 1$ et $j\neq n$.
\end{lem}

%Je comprends pas du tout le lemme qui suit.
%Que signifie la notation (T(u+a,v+c))_{1<=u<=b-a,1<=v<=d-c} ?
%Si i=a (resp j=c), on a i'=0 (resp j'=0). Dans ce cas, l'effondrement en (i',j') n'est pas défini, c'est bizarre
%-- Thomas

%Pourquoi noter (T(u+a,v+c))_{1<=u<=b-a,1<=v<=d-c} et pas (T(u,v))_{a<=u<=b,c<=v<=d}
%Ce qui est selon moi plus claire.
%Je fais la modif vous pouvez reverse si ça dérange des gens

%Pourquoi y-a-t-il toujours (T(u+a,v+c)) après modif ?
%-- Thomas

\begin{lem}
\label{lemmeextraction}
Soit $T,T'\in\mathcal{M}_{m,n}(\mb{N})$ tels que $T\xrightarrow{(i,j)}T'$. Soit $a,b,c,d\in\mb{N}$ tels que $0\leq a<b\leq m$ et $0\leq c<d\leq n$. Si $i\in \llb a,b\rrb$ et $j\in\llb c,d\rrb$ alors en notant $i'=i-a$, $j'=j-c$ on a 
$$(T(u,v))_{a\leq u\leq b, c\leq v\leq d} \xrightarrow{(i',j')} (T'(u,v))_{a\leq u\leq b, c\leq v\leq d}$$
\end{lem}

\begin{definition}[\bsc{Écoulement}]
Un écoulement infini est une suite de tas de sables $(T_k)_{k\in \mb{N}}$ telle que pour tout $l\in\mb{N}$, on a $T_l\xr{l}T_{l+1}$.

Un écoulement fini est une famille de tas de sables $(T_k)_{0\leq k\leq n}$ telle que pour tout $l\in\mb{N}$, avec $l<n$, on a $T_l\xr{l}T_{l+1}$.
\end{definition}

Les sections suivantes ont pour but de présenter diverses preuves de terminaison de la règle de l'effondrement. C'est-à-dire montrer qu'un tas de sable quelconque ne peut subir qu'un nombre fini d'effondrements avant que plus aucune de ses piles ne puisse s'effondrer (\textit{i.e.} obtenir un tas de sable dont tout les coefficients sont compris entre 0 et 3).


