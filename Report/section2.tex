\section{Une première preuve de terminaison (par une mesure)[Thomas,Remy]}

%Beaucoup de trucs pas très clairs dans cette partie
Dans cette section, nous allons montrer la terminaison de la règle de l'effondrement sur les tas de sables finis. Pour cela, nous allons construire une fonction de valuation, qui va associer à chaque tas de sable une valeur entière. On montrera ensuite que cette valeur décroit strictement à chaque effondrement.
%associer à chaque grain de sable un "poids" entier, qui dépendra de sa position dans la matrice. On veut faire en sorte que le "poids" total du tas de sable diminue strictement à chaque effondrement.

On définit pour $m,n \in \mb{N}$ la fonction $d_{m,n}$ de la manière suivante:

\[ d_{m,n} : \begin{array}{ccc}
       \mb{N}^2 & \longrightarrow& \mb{N}\\
      (i,j) & \longmapsto &  \min(i,j,m-i+1,n-j+1)
   \end{array}
\]

Cette fonction donne, pour la case (i,j) d'une matrice de taille $(m,n)$, la distance de la case au bord le plus proche.

Soit $a$ la fonction définie ainsi :
\[ a : \begin{array}{ccc}
       \mb{N} & \longrightarrow& \mb{N}\\
      i & \longmapsto &  \frac{i(i+1)}{2}
   \end{array}
\]
\begin{rem}
On notera que $a(i)$ n'est autre que la somme des entiers naturels compris entre 1 et $i$.
\end{rem}

Soit $T \in M_{m,n}(\mb{N})$ un tas de sable. On note $p = min(\lceil m/2 \rceil , \lceil n/2 \rceil)$, égal à la plus grande "distance au bord le plus proche de la matrice" des cases de $T$. Soit $w_{m,n}$ la fonction définie ainsi :

\[ w_{m,n} : \begin{array}{ccc}
       \mb{N} & \longrightarrow& \mb{N}\\
      d& \longmapsto &  a(p) + 1 - a(p - d) + p
   \end{array}
\]

%Autre possibilité: enlever le p de w_{m,n}(i,j) et remplacer la valuation par un mot de deux entiers: le nombre total de grains et la valeur de la valuation obtenue avec w_{m,n}(i,j) sans le +p.

En composant à gauche cette fonction avec $d_{m,n}$ on associe une valeur à chaque case de $T$ en créant des "couronnes concentriques", et où les valeurs des couronnes vont en croissant au fur et à mesure qu'on s'éloigne des bords. Voici un exemple des valeurs attribuées par $w_{m,n}\circ d_{m,n}$ :

%\[ \begin{cases}
 %     w_{m,n} : & \llbracket 1,m \rrbracket \times \llbracket 1,n \rrbracket \rightarrow \mb{N}\\
  %    w_{m,n}(i,j) = & a_{m,n}(p) - a_{m,n}(min(i,j,m-i+1,n-j+1))
   %\end{cases}
%\]

%\begin{center}
%\begin{tabular}{|c}
%$w_{m,n} : \llbracket 1,m \rrbracket \times \llbracket 1,n \rrbracket \rightarrow \mb{N}$\\
%$w_{m,n}(i,j)=a_{min(i,j,m-i+1,n-j+1)}$
%\end{tabular}
%\end{center}

\begin{center}
\begin{tabular}{|c|c|c|c|c|c|c|}
\hline
4&4&4&4&4&4&4\\
\hline
4&7&7&7&7&7&4\\
\hline
4&7&9&9&9&7&4\\
\hline
4&7&7&7&7&7&4\\
\hline
4&4&4&4&4&4&4\\
\hline
\end{tabular}
\end{center}
\begin{center}
Valeurs de $w_{m,n}\circ d_{m,n}$ sur une matrice de taille $5\times7$
\end{center}
\begin{rem}
Pour tous $m,n \in \mb{N}^2$, $w_{m,n}$ est strictement positive et strictement croissante sur l'ensemble des valeurs qui nous intéresse, c'est à dire l'intervalle $\llbracket 1,p \rrbracket$.
\end{rem}

On peut maintenant définir la valuation $v$ d'un tas de sable $T$ de taille $m\times n$ par:
%\[ \begin{cases}
	\[
    v(T) =  \sum\limits_{\substack{1 \leq i\leq m\\ 1 \leq j\leq n}} w_{m,n}\circ d_{m,n}(i,j)\times T(i,j)
	\]
    %\end{cases}
%\]

Cette valuation calcule la somme des poids des grains de sable du tas, le poids d'un grain étant calculée en appliquant la fonction $w_{m,n}\circ d_{m,n}$ à sa case.
\begin{rem}
Par construction, la valuation d'un tas de sable est toujours un entier positif.
\end{rem}
Montrons à présent que cette valuation est strictement décroissante par application d'un effondrement.

%<?>
%finir modification
%insérer schéma
%</?>

\begin{theo}
Soit $T,T'$ deux tas de sable finis. Si $T\xrightarrow{(i,j)}T'$, alors $v(T)>v(T')$.
\end{theo}

%TODO: Explique pourquoi ça fait fonctionner la preuve en général
\begin{proof}
On numérote les couronnes selon leur distance par rapport au bords de la matrice. Ainsi, la couronne numéro 1 est la couronne la plus à l'extérieur et contient toutes les cases à distance 1 des bords. Elle entoure la couronne numéro 2, qui entoure la couronne numéro 3 etc...\\
On notera $c$ la case de $T$ subissant l'effondrement et on notera $k$ le numéro de la couronne auquel appartient $c$.\\
On notera abusivement $w_{m,n}(k)$ pour désigner la valeur donnée par la fonction $w_{m,n}$ aux cases de la couronne $k$.\\
%<nécessaire?>
%On définit la "couronne inférieure" de la couronne composée des cases à distance $k$ des bords de la matrice comme étant la couronne composée des cases à distance $k-1$ des bords de la matrice. On définit de même la "couronne supérieure" de la couronne composée des cases à distance $k$ des bords de la matrice comme étant la couronne composée des cases à distance $k+1$ des bords de la matrice.
%</nécessaire?>

On va procéder par disjonction de cas. On identifiera ainsi les diverse configurations d'effondrement pouvant apparaître, pour lesquels on montrera que $v(T)-v(T')>0$
\begin{itemize}
\item{Si $p=1$ :}
	\begin{itemize}
	\item{Si $m=n=1$
        \begin{center}
        \begin{tabular}{|c|}
        \hline
        1\\
        \hline
        \end{tabular}
        \end{center}
        Ici, $c$ est adjacente aux quatre bords de la matrice. Cf configuration 1.
    }
	\item{Si ($m=1$ et $n=2$) ou ($m=2$ et $n=1$)
    	\begin{center}
    	\begin{tabular}{|c|c|}
        \hline
        1&1\\
        \hline
        \end{tabular}
        OU
     	\begin{tabular}{|c|}
        \hline
        1\\
        \hline
        1\\
        \hline
        \end{tabular}
		\end{center}
        Ici, $c$ est forcément adjacente à 3 bords et à une autre case ayant la même valeur. Cf configuration 2.
	}
    \item{Si ($m=1$ et $n>2$) ou ($m>2$ et $n=1$)
    	\begin{center}
        \begin{tabular}{|c|c|c|}
        \hline
        1&...&1\\
        \hline
        \end{tabular}
        OU
        \begin{tabular}{|c|}
        \hline
        1\\
        \hline
        \vdots\\
        \hline
        1\\
        \hline
        \end{tabular}
        \end{center}
        Ici, les cases aux extrémités sont dans la configuration 1, les cases centrales étant dans la configuration 3.
    }
    \item{Si $m=2$ et $n=2$
    	\begin{center}
        \begin{tabular}{|c|c|}
        \hline
        1&1\\
        \hline
        1&1\\
        \hline
        \end{tabular}
        \end{center}
        Ici, chacune des cases est dans la configuration 3.
    }
    \item{Si ($m=2$ et $n>2$) ou ($m>2$ et $n=2$)
    	\begin{center}
        \begin{tabular}{|c|c|c|}
        \hline
        1&...&1\\
        \hline
        1&...&1\\
        \hline
        \end{tabular}
        OU
        \begin{tabular}{|c|c|}
        \hline
        1&1\\
        \hline
        \vdots&\vdots\\
        \hline
        1&1\\
        \hline
        \end{tabular}
        \end{center}
    }
    Ici, les 4 cases présentes dans les coins sont dans la configuration 3. Les autres cases étant dans la configuration 4.
	\end{itemize}
\item{Si $p>1$ :}
Le fait que $T$ soit dans ce cas implique $m\geq 3$ et $n\geq 3$. Ceci entraîne la présence d'au moins 2 couronnes : une au centre et une au bord, avec potentiellement des couronnes intermédiaires.
La case $c$ peut donc être située soit sur la couronne extérieure, soit sur la couronne centrale, soit sur une couronne intermédiaire :
	\begin{description}
    	\item[$\bullet$]Si $c$ est sur la couronne extérieure :
        	\begin{center}
            \begin{tabular}{|c|c|cc|c|c|}
            \hline
            k&...&...&...&...&k\\
            \hline
            \vdots&k+1&...&...&k+1&\vdots\\
            \hline
            \vdots&\vdots&&&\vdots&\vdots\\
            \vdots&\vdots&&&\vdots&\vdots\\
            \hline
            \vdots&k+1&...&...&k+1&\vdots\\
            \hline
            k&...&...&...&...&k\\
            \hline
            \end{tabular}
            \end{center}
            Les cases dans les coins sont dans la configuration 3. Les autres cases de la couronne extérieure sont dans la configuration 5.
        \item[$\bullet$]Si $c$ est sur une couronne intermédiaire :
        	%ajouter une couronne extérieure en petits points
        	\begin{center}
            \begin{tabular}{|c|c|c|cc|c|c|c|}
            \hline
            k-1&...&...&...&...&...&...&k-1\\
            \hline
            \vdots&k&...&...&...&...&k&\vdots\\
            \hline
            \vdots&\vdots&k+1&...&...&k+1&\vdots&\vdots\\
            \hline
            \vdots&\vdots&\vdots&&&\vdots&\vdots&\vdots\\

            \vdots&\vdots&\vdots&&&\vdots&\vdots&\vdots\\
            \hline
            \vdots&\vdots&k+1&...&...&k+1&\vdots&\vdots\\
            \hline
            \vdots&k&...&...&...&...&k&\vdots\\
            \hline
            k-1&...&...&...&...&...&...&k-1\\
            \hline
            \end{tabular}
            \end{center}
            Les cases dans les coins de la couronne $k$ sont dans la configuration 6. Les autres cases de la couronne $k$ sont dans la configuration 7.
        \item[$\bullet$]Si $c$ est sur la couronne centrale :
        	La forme exacte de la couronne centrale dépend de la parité de $min(m,n)$ :
            \begin{description}
				\item[$\ast$]{Si $min(m,n)$ est pair :}
            		\begin{center}
                    	\begin{tabular}{|c|c|cc|c|c|}
                        	\hline
                        	k-1&...&...&...&...&k-1\\
                            \hline
                            \vdots&k&...&...&k&\vdots\\
                            \hline
                            \vdots&k&...&...&k&\vdots\\
                            \hline
                            k-1&...&...&...&...&k-1\\
                            \hline
                        \end{tabular}
                        OU
                        \begin{tabular}{|c|c|c|c|}
                        \hline
                        k-1&...&...&k-1\\
                        \hline
                        \vdots&k&k&\vdots\\
                        \hline
                        \vdots&\vdots&\vdots&\vdots\\
                        \hline
                        \vdots&k&k&\vdots\\
                        \hline
                        k-1&...&...&k-1\\
                        \hline
                        \end{tabular}
                    \end{center}
                    Les cases dans les coins de la couronne $k$ sont dans la configuration 6. Les autres case de la couronne $k$ sont dans la configuration 8.
            	\item[$\ast$]{Si $min(m,n)$ est impair :}
                	La forme de la couronne centrale est une ligne de longueur $max(m,n) - min(m,n) +1$. Cette ligne est verticale si $m\geq n$ est horizontale si $n\geq m$ (si $m=n$ la ligne est réduite à une case, qui est donc verticale et horizontale).
                    \begin{center}
                    \begin{tabular}{|c|c|cc|c|c|}
                    \hline
                    k-1&...&...&...&...&k-1\\
                    \hline
                    \vdots&k&...&...&k&\vdots\\
                    \hline
                    k-1&...&...&...&...&k-1\\
                    \hline
                    \end{tabular}
                    OU
                    \begin{tabular}{|c|c|c|}
                    \hline
                    k-1&...&k-1\\
                    \hline
                    \vdots&k&\vdots\\
                    \hline
                    \vdots&\vdots&\vdots\\
                    \hline
                    \vdots&k&\vdots\\
                    \hline
                    k-1&...&k-1\\
                    \hline
                    \end{tabular}
                    OU
                    \begin{tabular}{|c|c|c|}
                    \hline
                    k-1&...&k-1\\
                    \hline
                    \vdots&k&\vdots\\
                    \hline
                    k-1&...&k-1\\
                    \hline
                    \end{tabular}
                    \end{center}
                    Lorsque $m\neq n$, les cases situées aux extrémités de la couronne $k$ sont dans la configuration 9. Les autres cases de la couronne $k$ sont dans la configuration 6. Si $m=n$, alors l'unique case de la couronne $k$ est dans la configuration 10.
            \end{description}
	\end{description}
\end{itemize}
%Faire une liste de calculs et s'y référer dans la disjonction de cas. exemple d'un calcul: "effondrement d'une case de la couronne k entourée d'une case de la couronne k-1, d'une case de la couronne k+1 et de deux cases de la couronne k".

%SAUTE UNE LIGNE !!! SATANE LATEX !!!
%do medskip

Liste des configurations possibles pour la case $c$ et ses voisines, et calculs correspondants pour $v(T)-v(T')$ :
\begin{enumerate}
	\item{La case $c$ est adjacente aux quatre bords de la matrice :
    	%schema?
		\begin{eqnarray*}
		v(T)-v(T') &=& 4\times w_{m,n}(k) \\
        &>& 0
		\end{eqnarray*}
	}
    \item{La case $c$ est adjacente à 3 bords de la matrice, et à 1 case de la même couronne :
       	%schema?
    	\begin{eqnarray*}
		v(T)-v(T') &=& 4\times w_{m,n}(k) - w_{m,n}(k) \\
        &=& 3\times w_{m,n}(k)\\
        &>& 0
		\end{eqnarray*}
    }
    \item{La case $c$ est adjacente à 2 bords de la matrice, et à 2 cases de la même couronne :
    	%schema?
        \begin{eqnarray*}
		v(T)-v(T') &=& 4\times w_{m,n}(k) - 2\times w_{m,n}(k) \\
        &=& 2\times w_{m,n}(k)\\
        &>& 0
		\end{eqnarray*}
    }
    \item{La case $c$ est adjacente à 1 bords de la matrice, et à 3 cases de la même couronne :
     	%schema?
        \begin{eqnarray*}
		v(T)-v(T') &=& 4\times w_{m,n}(k) - 3\times w_{m,n}(k) \\
        &=& w_{m,n}(k)\\
        &>& 0
		\end{eqnarray*}
    }
    \item{La case $c$ est adjacente à 1 bord, à 2 cases de la couronne extérieure ($k=1$) et à 1 case de la couronne 2 ($=k+1$).
        \begin{eqnarray*}
		v(T)-v(T') &=& 4\times w_{m,n}(k) - 2\times w_{m,n}(k) - w_{m,n}(k+1)\\
        &=& 2\times w_{m,n}(k) - w_{m,n}(k+1)\\
        &=& 2\times w_{m,n}(1) - w_{m,n}(2)\\
        &=& 2\times(a(p) + 1 - a(p-1) + p) - (a(p) + 1 - a(p-2) + p)\\
        &=& a(p) + 1 - 2\times(a(p-1)) + a(p-2) + p\\
        &=& p + 1 - (p-1) + p\\
        &=& p + 2\\
        &>& 0
		\end{eqnarray*}
    }
    \item{La case $c$ est adjacente à 2 cases de la même couronne (la couronne $k$) et à 2 cases de la couronne $k-1$ :
    %schema?
    \begin{eqnarray*}
    v(T)-v(T') &=& 4\times w_{m,n}(k) - 2\times w_{m,n}(k) - 2\times w_{m,n}(k-1)\\
    &=& 2\times w_{m,n}(k) - 2\times w_{m,n}(k-1)\\
    &=& 2\times(a(p) + 1 - a(p-k) + p) - 2\times(a(p) + 1 - a(p-(k-1)) + p)\\
    &=& -2a(p-k) + 2a(p-(k-1))\\
    &>& 0
	\end{eqnarray*}
    }
    \item{La case $c$ est adjacente à 2 cases de la même couronne (la couronne $k$), à 1 cases de la couronne $k-1$ et à 1 case de la couronne $k+1$ :
    %schema?
    \begin{eqnarray*}
    v(T)-v(T') &=& 4\times w_{m,n}(k) - 2\times w_{m,n}(k) - w_{m,n}(k-1) - w_{m,n}(k+1)\\
	&=& 2\times w_{m,n}(k) - w_{m,n}(k-1) - w_{m,n}(k+1)\\
    &=& 2\times(a(p) + 1 - a(p-k) + p) - (a(p) + 1 - a(p-(k-1)) + p)\\
    && - (a(p) + 1 - a(p-(k+1)) + p)\\
    &=& -2a(p-k) + a(p-(k-1)) + a(p-(k+1))\\
    &=& p - (k-1) - (p-k)\\
    &=& 1\\
    &>& 0
	\end{eqnarray*}
    }
    \item{La case $c$ est adjacente à 3 cases de la même couronne (la couronne $k$), et à 1 case de la couronne $k-1$ (on notera que p est par définition supérieur ou égal à tout numéro de couronne) :
    %schema?
    \begin{eqnarray*}
    v(T)-v(T') &=& 4\times w_{m,n}(k) - 3\times w_{m,n}(k) - w_{m,n}(k-1)\\
    &=& w_{m,n}(k) - w_{m,n}(k-1)\\
    &=& a(p) + 1 - a(p-k) + p - (a(p) + 1 - a(p-(k-1)) + p)\\
    &=& -a(p-k) + a(p-(k-1))\\
    &=& p-(k-1)\\
    &=& p - k + 1\\
    &>& 0
	\end{eqnarray*}
    }
    \item{La case $c$ est adjacente à 1 case de la même couronne (couronne $k$) et à 3 cases de la couronne $k-1$ :
    %schema?
    \begin{eqnarray*}
    v(T)-v(T') &=& 4\times w_{m,n}(k) - w_{m,n}(k) - 3\times w_{m,n}(k-1)\\
    &=& 3\times (a(p) + 1 - a(p-k) + p) - 3\times(a(p) + 1 - a(p-(k-1)) + p)\\
    &=& -3a(p-k) + 3a(p-(k-1))\\
    &=& 3(p-(k-1))\\
    &>& 0
	\end{eqnarray*}
    }
    \item{La case $c$ est adjacente à 4 cases de la couronne $k-1$ :
    \begin{eqnarray*}
    v(T)-v(T') &=& 4\times w_{m,n}(k)) - 4\times w_{m,n}(k-1)\\
    &=& 4\times(a(p) + 1 - a(p-k) + p) - 4\times(a(p) + 1 - a(p-(k-1)) + p)\\
    &=& -4a(p-k) + 4a(p-(k-1))\\
    &=& 4(p-(k-1))\\
    &>& 0
	\end{eqnarray*}
    }
\end{enumerate}





%On note $c$ la case (i,j).
%Les seuls cases qui vont voir leurs nombre de grain de sable changer au cours de l'effondrement sont les cases adjacentes à $c$ et $c$ elle-même. On omettra donc dans les calculs de $v(T)-v(T')$ qui vont suivre de considérer les autres cases (leurs apport à la valuation de $T$ et à celle de $T'$ étant identique et s'annulant).
%Considérons toute les positions que la case $c$ est susceptible d'occuper :
%\begin{itemize}
%\item $c$ est sur la couronne centrale (celle qui a la valeur par $w_{m,n}$ la plus élevée). Il n'y a que deux configurations possibles pour la couronne centrale. Il s'agit soit d'une ligne verticale (respectivement horizontale) de 3 cases si m
%\end{itemize}

%L'effondrement de la case (i,j) (notée $c$ dans la suite) ne va modifier le no
%Étant donné que le nombre de grains de sable ne va pas changer ailleurs dans le tas, concentrons nous sur la case $(i,j)$ (notée $c$) et les cases lui étant adjacentes. On distingue deux cas :
%\begin{itemize}
%\item $c$ est adjacente à exactement une case (notée $c_-$) ayant un poids (image par $w_{(m,n)}$) strictement inférieur. Dans ce cas, $c$ est également adjacente à deux autres cases $c_1$ et $c_2$ ayant le poids $w_{(m,n)}(c)$, et la quatrième case (notée $c_+$) a soit un poids strictement supérieur, soit n'existe pas (si $c$ est sur le bord de la matrice). On effectue le calcul de la différence des valuations de $T$ et $T'$ en supposant que $c_+$ existe:

%\begin{eqnarray*}
%v(T)-v(T') &=& 4\times w_{(m,n)}(c) - w_{(m,n)}(c_-) - w_{(m,n)}(c_1) - w_{(m,n)}(c_2) - w_{(m,n)}(c_+)\\
%&=& 4\times w_{(m,n)}(c) - w_{(m,n)}(c_-) - 2w_{(m,n)}(c) - w_{(m,n)}(c_+)\\
%&=& 2\times w_{(m,n)}(c) - w_{(m,n)}(c_-) - w_{(m,n)}(c_+)\\
%&=& 2\times a_{(m,n)}(p) - 2\times a_{(m,n)}(d) - a_{(m,n)}(p)\\
%& & + a_{(m,n)}(d-1) - a_{(m,n)}(p) + a_{(m,n)}(d+1)\\
%&=& -2\times a_{(m,n)}(d) + a_{(m,n)}(d-1) + a_{(m,n)}(d+1)\\
%&>& 0
%\end{eqnarray*}
%La non-existence de $c_+$ facilite les calculs mais ne change pas le résultat.

%\item $c$ n'est adjacente à aucune case ayant une valeur strictement inférieure ($c$ est sur le coin d'une couronne).

%\item Essai nb 2
%\item $c$ est sur la couronne centrale (i.e. $c$ a un poids de 0).
%\item $c$ est sur le coin d'une couronne, à l'exception de la couronne centrale.
%\item $c$ est sur une couronne à l'exception de la couronne centrale, mais pas sur un coin.



%\end{itemize}
%On remarque aisément qu'une case $c$ est adjacente à au plus u

%\begin{itemize}
%\item Si $(i,j) \in {(1,1),(1,n),(m,1),(m,n)}$ : la case s'effondrant est sur l'un des coins du tas de sable, dans ce cas, on a : $v(T)-v(T')= 4*1 - 2*1 = 2 > 0$.
%\item
%\end{itemize}
% Dans la mesure où cette preuve ne concerne qu'un seul effondrement, on peut, pour plus de simplicité, étendre les matrices $T$ et $T'$, ainsi que la fonction $w_{(m,n)}$ à l'ensemble des couples d'entiers naturels, en leur attribuant une valeur de 0 sur les couples hors de leurs domaine de définition précédent.% (par exemple $T(1,n)$) comme ayant une valeur nulle par la fonction $w_{(m,n)}$ (suite de l'exemple : $w_{(m,n)}(1,n)=0$).


\end{proof}

La proposition 2.4 permet d'établir la terminaison . En effet, si il existait un écoulement infini, la suite des valuations des tas de sable associé serait une suite d'entiers naturels positif strictement décroissante, ce qui est impossible.

\medskip

Cette preuve nous donne également une borne sur le nombre d'effondrement nécessaire pour arriver à un tas de sable non effondrable.

\begin{theo}
Soit $T$ un tas de sable fini de taille $m \times n ( m > n )$ et dont les coefficients sont bornés par $t_0$. La taille d'un écoulement partant de T sera en $\mathrm{O} ( t_0 \times m^3 \times )$
\end{theo}

\begin{proof}

La valuation de la matrice T vérifie:

	\[
    v(T) =  \sum\limits_{\substack{1 \leq i\leq m\\ 1 \leq j\leq n}} w_{m,n}\circ d_{m,n}(i,j)\times T(i,j)
	\]

    \[
    v(T) < t_0 \times \sum\limits_{\substack{1 \leq i\leq m\\ 1 \leq j\leq n}} w_{m,n}(\lceil m/2 \rceil)
    \]

\end{proof}

