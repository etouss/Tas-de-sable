\section{Généralisation à des graphes finis}
%cela s'adapte à des graphes non-connexes orienté avec arête multiple ??
%Non connexes ne sert pas car cela reviens à considérer n problèmes connexes.
On considère ici des graphes connexes non-orientés à arête simple $G=(V,E)$ où $V$ est l'ensemble des noeuds et $E$ l'ensemble des arête. On dit que $v'$ est un voisin de $v$ si $\{v,v'\}\in E$. On note $d(v)$ le degré d'un noeud $v$, c'est-à-dire son nombre de voisins.

\begin{definition}[\bsc{Graphe de sable}]
Un graphe de sable $T$ est un graphe fini $G=(V,E)$ muni d'une fonction $g:V\ra\mb{N}$ dites nombre de grains de sables et d'une fonction $e:V\ra\mb{N}$ dites fonctions d'écoulement.
\end{definition}

\begin{definition}[\bsc{Effondrement}]
On définit une relation binaire noté $\xra{v}$ sur les graphes de sables. $T=((V,E),g,e)$ et $T'=(G',g',e')$ sont en relation $T\xra{v}T'$ si et seulement si $G=G'$, $e=e'$ et de plus $\forall x\in V, g(x)=g'(x)$ sauf dans les cas suivants :
\begin{itemize}
\item Si $x=v$ alors $g'(v)=g(v)-d(v)-e(v)$.
\item Si $x$ est un voisin de $v$ alors $g'(x)=g(x)+1$
\end{itemize}
\end{definition}

\begin{rem}
Cette définition implique $g(v)\geq e(v)+d(v)$.
\end{rem}

\begin{definition}
Soit $T=((V,E),g,e)$ un graphe de sable.

On dit que $T$ a une fuite si $\exists v\in V, e(v)>0$.

On appelle le nombre de grain de  $T$ l'entier naturel $\sum_{v\in V} g(v)$, noté $|T|$.
\end{definition}

On présente ici des lemmes qui découlent de la définition, qui seront utiles plus tard.
\begin{lem}
\label{lemmedecroissancegraph}
Soit $T=(G,g,e)$ et $T'=(G,g',e)$ des graphes de sable tels que $T\xra{v}T'$. Alors $|T|\geq |T'|$. 

De plus $|T|=|T'|$ si et seulement si $e(v) = 0$.
\end{lem}

\begin{lem}
\label{lemmeextractiongraph}
Soit $T=((E,V),g,e)$ et $T'=((E,V),g',e)$ des graphes de sable tels que $T\xra{v}T'$. Soit $V_1 \subseteq V $. \\
Si $v\in V_1$ alors en notant $E_1 = E \cap (V_1 \times V_1  )$ et en posant pour tout $v\in V_1$, $e_1(v)=e(v) + |\{v\in V \setminus V_1\}|$, $T_1 = ((E_1,V_1),g_{|V_1},e_1)$ et $ T_1'= ((E_1,V_1),g'_{|V_1},e_1) $ on a 
\begin{equation*} T_1 \xra{v} T_1' \end{equation*}
\end{lem}


\begin{rem}
On définit d'ailleurs les écoulement de graphe de sable (fini et infini) de manière analogue.

La confluence forte des graphes de sables se montre aussi de même.
\end{rem}

\begin{lem}
\label{lemmeecoulementinfini}
 Soit $T$ un graphe de sable. Si il existe un écoulement infini alors cette écoulement effondre une infinité de fois chaque sommet de $V$.
\end{lem}

\begin{proof}
Soit $T$ un graphe de sable qui possède un écoulement infini $T_0\xra{v_0}T_1\xra{v_1} \cdots \xra{v_{k-1}} T_k \xra{v_k} \cdots$\\
On pose $V'=\{v\in V | \forall n \in \mathbb{N}, \exists k > n \text{ tel que } v = v_k\}$ l'ensemble des sommets qui sont effondrer une infinité de fois dans un écoulement infini.\\
Si $V'\neq V$, alors on trouve $v\in V\setminus V'$ tel que $v$ appartienne au voisinage d'un sommet $v'$ des sommets de $V$. Or comme $v'$ est effondré un infinité de fois, on trouve une extractrice $\phi$ tel que, pour tout $k, T_k \xra{v'} T_{k+1}$. Ainsi par définition, pour tout $k$, $g_{\phi(k+1)}(v) = g_{\phi(k)}(v) + 1$ or d'après le lemme \ref{lemmedecroissancegraph} $g_{k}(v)\leq|T_0|$.\\
Absurde.\\
Donc $V'=V$ et un écoulement infini effondre une infinité de fois chaque sommet de $V$.
\end{proof}

\begin{theo}
Soit $T$ un graphe de sable. Si $T$ a une fuite, alors il n'existe pas d'écoulement infini de graphe de sables partant de $T$.
\end{theo}
\begin{proof}
% * <mathieu.huot@voila.fr> 2015-09-28T12:33:03.707Z:
%
% 
%
La terminaison se démontre de façon similaire, par récurrence sur la taille des graphes. De même on obtient un écoulement cyclique où $|T|$ se conserve (on remarque que si $T$ a une fuite et $T\ra^*T'$ alors $T'$ a une fuite). Ainsi les cases qui ont une fuite ne s'effondre jamais, et on peut extraire un sous-graphe de sable avec un écoulement cyclique. Ce sous-graphe a aussi une fuite car par connexité du graphe on a forcément supprimé un voisin en extrayant un sous-graphe et donc ajouté une fuite, d'où une contradiction.
\end{proof}

Autre preuve:
\begin{proof}
Conséquence direct du lemme précédant.
\end{proof}
\section{Généralisation à des graphes infinis}
%je n'ai jamais manipulé de graphe infini, j'ai mis des conditions pour éviter des graphes bizarres, mais je ne sais pas si elle sont nécessaires, et pas certain qu'elles soient suffisantes.
On considère ici des graphes $G=(V,E)$ infinis dénombrables, connexes non-orientés à arêtes simples. On suppose que le nombre de voisins est fini pour tout noeud. On suppose que le nombre de grains de sable est fini, c'est-à-dire $\sum_{v\in V} g(v)<\infty$. On suppose de plus que le graphe est sans perte. On souhaite montrer la terminaison. %Par connexe on entend qu'il existe une chemin fini entre chacun de ses noeuds.
%par définition il me semble qu'un graphe ne possède pas de chemin infini entre 2 noeuds.

\begin{lem}
\label{fuiteinfini}
Soit $T_0$ un graphe de sable infini. Supposons un écoulement infini $T_0\xra{v_0}T_1\xra{v_1}\cdots$. Alors $\{v_i|i\in\mb{N}\}$ est infini.
\end{lem}
\begin{proof}
Sinon, on peut extraire le graphe fini des $\{v_i|i\in\mb{N}\}$ en remplaçant les voisins supprimés par des fuites. Chacune de ses composantes connexes a une fuite sinon le graphe de départ a une composante connexe fini et n'est donc pas connexe. Ainsi on est ramené au cas fini.
\end{proof}

\begin{lem}
\label{consistancesommets}
Soit $T$ un graphe de sable infini, soit $u$ et $v$ deux nœuds voisins. Si $g(u)+g(v)>0$ alors pour tout $T'$ tel que $T\ra^* T'$, on a $g(u)+g(v)>0$.
\end{lem}
\begin{proof}
En résonant sur le dernier effondrement qui concerne $u$ ou $v$.
Soit $u$ est effondré donc $g(v)\geq 1$, soit $v$ est effondré donc $g(u) \geq 1$.
D'où $g(u) + g(v) \geq 1$.\\
Si ni $u$, ni $v$ ne sont effondrés alors $g(u) + g(v) > 0$.
\end{proof}

\begin{theo}
Il ne peut y avoir d'écoulement infini de graphes de sable infini.
\end{theo}

\begin{proof}
Preuve par l'absurde.\\
Si il existe un écoulement infini $T_0\xra{v_0}T_1\xra{v_1}\cdots$, alors d'après le lemme \ref{fuiteinfini} $\{v_i|i\in \mathbb{N}\}$ est infini. Donc (à développer), il y a un chemin infini dans le graphe qui est composé de sommet effondré par l'écoulement. Or en découpant le chemin par couple de 2 sommets, d'après le lemme \ref{consistancesommets} il y a une infinité de grain de sable, ce qui est absurde.
\end{proof}
%pseudo preuve il faut que je la reprenne.

